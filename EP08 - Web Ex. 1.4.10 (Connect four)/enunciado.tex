\documentclass{article}

\usepackage{enumerate}
\usepackage{hyperref}
\usepackage[T1]{fontenc}
\usepackage[utf8]{inputenc}
\usepackage{listings}
\usepackage{xcolor}

\lstset{%
  language=[LaTeX]TeX,
  backgroundcolor=\color{gray!25},
  basicstyle=\ttfamily,
  breaklines=true,
  columns=fullflexible
}

\title{Web Ex. 1.4.10 (Connect four)}
\date{}
\author{}

\begin{document}

\maketitle

Fa\c{c}a o Web Exercise 1.4.10 (Connect four) de IntroCS:
\bigbreak
\url{http://introcs.cs.princeton.edu/java/14array/}
\bigbreak

\begin{enumerate}[i]
	\item Inicialmente, voc\^e deve escrever o programa ConnFourGenerator.java para gerar configura\c{c}\~oes aleat\'orias N por N de '*', 'X' e 'O', com cada s\'imbolo aparecendo com probabilidade 1/3. Seu programa deve receber como argumento o valor de N. Seu programa deve tamb\'em receber um argumento adicional (0 ou 1), que indica se queremos a sa\'ida na forma tabular ou linear, como nos exemplos abaixo:
	\begin{lstlisting}[breaklines, gobble=16]
		$ java ConnFourGenerator 4 0
		* * O O 
		O * X * 
		O O O O 
		O X * * 
		$ java ConnFourGenerator 4 1
		*OXX*OXOX*X*O**O
		$ java ConnFourGenerator 10 0
		* X O * O * X X * X 
		O * X O O O X * * O 
		X X * X O * X X * * 
		O O O X O X X * X O 
		X X X O O X O X * X 
		X O X O * O O X X X 
		X O X X O X * * X X 
		X O O * * * X O * O 
		* * X * * X O X * O 
		O X O X O X * * X * 
		$ java ConnFourGenerator 10 1
		O*XX**OX*X****O*XOX**O**OXOO*XO**OOOXXXOXXOO*OXXXOO*OXOX*OOXX*OOXO*XXXOX*OXO*O*O**XO*OXO*XXO**XXO*OX
	\end{lstlisting}
	\item Voc\^e deve escrever o programa ConnFour.java que recebe uma configura\c{c}\~ao N por N de '*', 'X' e 'O' como argumento e encontra uma sequ\^encia cont\'igua mais longa de s\'imbolos do mesmo tipo, como especificado no enunciado do Web Ex. 1.4.10.
	\begin{lstlisting}[breaklines, gobble=16]
		$ java ConnFour OOXX*XOOOXOXOXXO
		O O X X 
		* X O O 
		O X O X 
		O X X O 
		3 / (1, 1) [X / v]
		$ java ConnFour O*XOXOO**XO*OX*O*OOX*O*O**OXXX*X*OX*OOOXXOOOO*XXO**O*O*O*O*XOO*XOOXX*OOXX*XOOOXOXOXXOXXXXXXX*XOXXXX*
		O * X O X O O * * X 
		O * O X * O * O O X 
		* O * O * * O X X X 
		* X * O X * O O O X 
		X O O O O * X X O * 
		* O * O * O * O * X 
		O O * X O O X X * O 
		O X X * X O O O X O 
		X O X X O X X X X X 
		X X * X O X X X X * 
		5 / (8, 5) [X / h]
	\end{lstlisting}
	No primeiro exemplo acima, a sequ\^encia encontrada tem comprimento 3, começa na entrada (1, 1), \'e uma sequ\^encia de X e \'e vertical. Analogamente, no segundo exemplo, a sequ\^encia encontrada tem comprimento 5, come\c{c}a na entrada (8, 5), \'e uma sequência de X e \'e horizontal.
\end{enumerate}

\end{document}