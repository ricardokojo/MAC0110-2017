\documentclass{article}

\title{Enunciado - Ex. 0 (Qualquer mensagem)}
\date{}
\author{}

\begin{document}

Lembre o exemplo visto em sala:\\

Alice quer enviar a mensagem SENDMONEY para Bob.

Alice e Bob t\^em a chave comum yT25a5i/S.

Usando o algoritmo de one-time pad, Alice produz o texto cifrado gX76W3v7K, que ela envia para Bob.

Eve consegue acesso \`a mensagem cifrada gX76W3v7K. Ela tenta todas as chaves poss\'iveis para decriptar esta mensagem cifrada. Entretanto, ao tentar a chave tTtpWk+1E, ela obt\'em o texto NEWTATTOO (veja p\'agina 19 da aula "Prologue: A Simple Machine"). Assim, ela poderia pensar (incorretamente) que a mensagem que Alice enviou a Bob \'e NEWTATTOO.

	\begin{enumerate}
		\item Mostre que h\'a uma chave que, quando usada por Eve, gera o texto ILikeEve+.
		\item Mostre tamb\'em que existe uma chave que gera o texto IHateEve+.
		\item Mais geralmente, prove que, para qualquer texto de 9 caracteres da tabela Base64 (página 9 das transparências), h\'a uma chave que a gera.
		\item B\^onus: encontre as chaves que geram ILikeEve+ e IHateEve+.
	\end{enumerate}

\end{document}