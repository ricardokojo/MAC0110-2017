\documentclass{article}

\title{Ex. 3 (Amigo secreto novamente)}
\date{}
\author{}

\begin{document}

\maketitle

Os N amigos do Ex. 2 (Amigo secreto) gostam da brincadeira e resolvem continuar suas investiga\c{c}\~oes. Para o amigo secreto, \'e inconveniente que haja na permuta\c{c}\~ao a[] um \'indice tal que a[i] = i (por qu\^e?).
\bigbreak
Seja $q_N$ a probabilidade de uma permuta\c{c}\~ao aleat\'oria dos n\'umeros 0, ... , N - 1 n\~ao ter um tal \'indice i. Escreva um programa que estima $q_N$.
\bigbreak
Mais precisamente, seu programa deve receber como entrada inteiros $NMAX$ e $T$. Seu programa deve ter como sa\'ida as estimativas obtidas para $q_2$, $q_3$, ..., $q_{NMAX}$. Para obter cada $q_N$, seu programa deve gerar T permuta\c{c}\~oes aleat\'orias no processo de estima\c{c}\~ao.
\bigbreak
\textbf{B\^onus.} Intua uma f\'ormula fechada para $q_N$ para N grande.
\bigbreak
Levando em conta suas descobertas, escreva um segundo programa, digamos PermutacoesBoas.java, que gera permutações de 0, ..., N - 1 que n\~ao t\^em tais \'indices i. Note que queremos que as permuta\c{c}\~oes geradas n\~ao tenham nenhum vi\'es!
\bigbreak
\textbf{B\^onus.} Escreva um aplicativo para celulares para gerar tais permuta\c{c}\~oes e tente vend\^e-lo para o grupo dos N amigos sorriso

\end{document}