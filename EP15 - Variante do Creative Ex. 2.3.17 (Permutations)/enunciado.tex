\documentclass{article}

\usepackage[T1]{fontenc}
\usepackage[utf8]{inputenc}
\usepackage{listings}
\usepackage{xcolor}

\lstset{%
  language=[LaTeX]TeX,
  backgroundcolor=\color{gray!25},
  basicstyle=\ttfamily,
  breaklines=true,
  columns=fullflexible
}

\title{Variante do Creative Ex. 2.3.17 (Permutations)}
\date{}
\author{}

\begin{document}

\maketitle

Estude o Creative Exercise 2.3.17, Permutations, de IntroCS. Escreva uma variante de Permutations.java chamado Arrangements.java, que recebe um string na linha de comando e imprime todos os rearranjos do string dado, sem repeti\c{c}\~ao. O string dado poder\'a ter as letras a..z e A..Z.
\bigbreak
Exemplos:
\begin{lstlisting}[breaklines, gobble=8]
	$ java-introcs Arrangements abc
	abc
	acb
	bac
	bca
	cab
	cba
	$ java-introcs Arrangements aaa
	aaa
	$ java-introcs Arrangements aab
	aab
	aba
	baa
	$ java-introcs Arrangements aabb
	aabb
	abab
	abba
	baab
	baba
	bbaa
	$ java-introcs Arrangements ABBCCC
	ABBCCC
	ABCBCC
	ABCCBC
	ABCCCB
	ACBBCC
	ACBCBC
	ACBCCB
	ACCBBC
	ACCBCB
	ACCCBB
	...
	(esta execução de Arrangements gera 60 strings de saída.)
\end{lstlisting}
\bigbreak
\textbf{Observa\c{c}\~ao.} Seu programa deve ter tempo de execução basicamente proporcional ao número de caracteres na saída (supondo que o string de entrada tem comprimento limitado, isso equivale a dizer que o tempo de execução de seu programa é basicamente proporcional ao números de strings impressos na saída).
\bigbreak
\textbf{Sugest\~ao.} Voc\^e pode achar interessante usar um vetor booleano "indexado por caracteres".  Em Java, o tipo char \'e basicamente equivalente a um inteiro (mais precisamente, um inteiro curto sem sinal (16 bits)).  A tradu\c{c}\~ao entre caracteres e inteiros \'e feita usando-se uma tabela de convers\~ao padr\~ao.  Os programas anexos d\~ao dicas de como manipular vetores indexados por chars.

\end{document}