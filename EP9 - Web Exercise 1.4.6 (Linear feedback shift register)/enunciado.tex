\documentclass{article}

\usepackage{enumerate}
\usepackage{hyperref}
\usepackage[T1]{fontenc}
\usepackage[utf8]{inputenc}
\usepackage{listings}
\usepackage{xcolor}

\lstset{%
  language=[LaTeX]TeX,
  backgroundcolor=\color{gray!25},
  basicstyle=\ttfamily,
  breaklines=true,
  columns=fullflexible
}

\title{Web Exercise 1.4.6 (Linear feedback shift register)}
\date{}
\author{}

\begin{document}

\maketitle

Fa\c{c}a a seguinte variante do Web Exercise 1.4.6 (Linear feedback shift register). Seu programa LFSRGen.java deve receber na linha de comando a "semente" e o "tap" a serem usados em seu LFSR. Al\'em disso, um terceiro argumento deve especificar quantos bits devem ser gerados. Exemplos de execu\c{c}\~ao:
\bigbreak
\begin{lstlisting}[breaklines, gobble=8]
	$ java LFSRGen 01101000010 8 40
	1100100100111101101110010110101110011000
	$ java LFSRGen 01101000010 8 5
	11001
	$ java LFSRGen 01110001010 8 40
	1011010000001100100001111101001100010011
	$ java LFSRGen 01110001010 7 40
	1111101110100100110100000100101001000011
\end{lstlisting}
\bigbreak
Este exercício \'e uma parte inicial do exerc\'icio em

\url{http://www.cs.princeton.edu/courses/archive/spr15/cos126/assignments/lfsr.html}

que você far\'a mais para frente.
\bigbreak
\textbf{Observa\c{c}ão.} Para saber qual \'e o i-\'esimo caractere de um string s, use s.charAt(i). Veja o programa anexo QuantosAs.java.
\bigbreak
Veja tamb\'em \url{http://introcs.cs.princeton.edu/java/11cheatsheet/}
\end{document}