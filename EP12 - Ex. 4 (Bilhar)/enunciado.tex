\documentclass{article}

\usepackage{enumerate}

\title{Ex. 4 (Bilhar)}
\date{}
\author{}

\begin{document}

\maketitle

Estude os programas BouncingBall.java, MouseFollower.java e OneSimpleAttractor.java. Escreva uma variante de BouncingBall.java, digamos Billiards.java, como descrito a seguir.
\bigbreak
\begin{enumerate}
\item Billiards.java deve ser tal que a bola sofre atrito e perde velocidade ao longo do tempo (veja a implementa\c{c}\~ao de "drag" em OneSimpleAttractor.java).
\item Billiards.java deve ficar escutando o mouse. Quando o mouse \'e pressionado, uma força deve agir sobre a bola, acelerando-a. Essa for\c{c}a deve ser na dire\c{c}\~ao da bola, a partir da posi\c{c}\~ao do mouse. Assim, essa for\c{c}a deve impulsionar a bola para longe da posi\c{c}\~ao do mouse.
\item A for\c{c}a que age na bola ao se pressionar o mouse deve ter intensidade inversamente proporcional ao quadrado da dist\^ancia entre o mouse e a bola. Assim, quando a bola est\'a longe do mouse, pressionar o mouse tem um efeito mais reduzido. Se o mouse est\'a bem perto da bola, ao se pressionar o mouse, a bola sofre uma for\c{c}a grande e ela sofre uma acelera\c{c}\~ao grande.
\end{enumerate}
\bigbreak
O pequeno v\'ideo anexo mostra o comportamento desejado de Billiards.java. Para que fique claro quando o mouse foi pressionado, neste v\'ideo o fundo da imagem torna-se branco quando o usu\'ario pressiona o mouse.
\bigbreak
\textbf{Desafio:} quando seu programa ficar pronto e estiver rodando bem, tente perder pouco tempo brincando com ele :)

\end{document}